\section{总结与收获}

本次实践不仅能帮助其他同学更轻松地完成大物实验报告,我们自己也受益良多。

\begin{description}
  \item[分工与合作] 我们分工明确,每个人的任务都有截止时间,这使我们小组的进度有序推进。以往的经历中,代码与相关工作往往都是独立完成,代码规范与项目进程完全由自己安排。但是在这种大工程中,相关代码需要符合规范,需要与队友交接,工作进度也要与队友进度相符。在这种分工体系下,每个人都要完成自己的任务,并顾及与他人的交互。
  \item[代码规范性] 本次大作业中,我们建立了统一的码风,并制定了自主编写的 API 的使用说明。这样做一方面可以使得产品最终具有一致性——不同人写的代码能够基本一致;另一方面也使得最终的检验与调整能够更加方便——规范的代码提高了代码的可读性,降低了代码的审核成本。
  \item[软件开发技巧] 在本次实践中,我们使用 git 进行协作,代码注释清楚,帮助文档详细。这大大提高了我们的开发效率。
\end{description}

接下来,我们计划将本项目开源,并增添诸如“网页上输入数据”“手写实验数据 OCR”等功能。
更进一步地,我们会考虑支持二级大物实验,以及将本产品推广至其他院校。