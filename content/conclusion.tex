\section{总结与展望}
\subsection{项目源码}
大雾实验工具源码和开发文档见\href{https://github.com/feixukeji/PhyX}{GitHub},蜗壳排课工具源码见\href{https://rec.ustc.edu.cn/share/58855c20-6911-11ee-946f-d117a7db9848}{此处},我的科大APP源码见\href{https://rec.ustc.edu.cn/share/1e33e8c0-6911-11ee-bd5b-631c3fa07e2e}{此处}。

\subsection{产品鲁棒性}
我们在开发过程中采取了有效的技术手段,使产品具有极佳的鲁棒性。无论胡乱输入什么东西,或者有意做任何非常规操作,我们的应用均不会奔溃。

\subsection{蜗壳作业助手}
除了上述三款已发布的产品外,我们还在筹划一款新的产品——蜗壳作业助手。

众所周知,我校布置作业的平台众多,各课程所用平台各异。若没能记住,便需要在QQ群消息、群文件、群公告、群作业、BB平台、瀚海教学网等各处寻觅。并且,作业完成数日后,又会忘记自己是否已完成、是否已提交,于是不得不又前往各个平台检查。

我们痛定思痛,计划打造一款统一的作业布置、提交平台,已完成和未完成的作业分别列表呈现。

不仅如此,我们将利用API自动获取以上各个平台的作业信息,使同学们免于繁琐的翻阅。当学生完成作业后,对于需要提交电子版的作业,直接提交在蜗壳作业助手上,助手将自动通过API同步提交到上述平台;对于需要上交纸质版的作业,可以在助手上做打钩标记,以便使自己记得确实完成了该作业。

我们已经完成了程序接口的设计,即将开始实现相应功能。

\subsection{生态融合}
我们这几款产品并非独立的产品,我们正积极推进生态融合。目前,大雾实验工具和蜗壳排课工具已集成至我的科大APP,这是生态融合的初步尝试。未来,我们将进一步融合,例如将我的科大APP的“任务清单”功能与蜗壳作业助手进行结合。

\subsection{团队协作}
我们分工明确,使用 git 进行协作,每个人的任务都有截止时间,这使我们的进度有序推进。以往的经历中,代码与相关工作往往都是独立完成,代码规范与项目进程完全由自己安排。但是在这种大工程中,相关代码需要符合规范,需要与队友交接,工作进度也要与队友进度相符。在这种分工体系下,每个人都要完成自己的任务,并顾及与他人的交互。

特别是大雾实验工具,我们建立了统一的码风,代码注释清楚,并制定了自主编写的 API 的详细使用说明。这样做一方面可以使得产品最终具有一致性——不同人写的代码能够基本一致;另一方面也使得最终的检验与调整能够更加方便——规范的代码提高了代码的可读性,降低了代码的审核成本。

\subsection{期待合作}
我们深知“一花独放不是春,百花齐放春满园”,“蜗壳365”秉持合作开放的态度,热切期待与其他同学的合作。

\section{致谢}
首先特别感谢余庆杯组委会为我们提供这样一个项目开发和展示的机会。

其次,在“蜗壳365”的开发与运维过程中,我们在视觉艺术、功能模块、宣传推广、意见反馈等方面得到了许多同学的帮助,列举如下:

苏宗山、秦沁、鲍政廷、周旭冉、尹冠霖、夏熙林、陈艺雨、陈思、王星河、蔡卓凡、施耀炜、李昊、邱梓惠、陈逸翀……

在此特别向他们表示衷心的感谢。如果获奖,我们会将奖金分予他们以表谢意。

同时,我们感谢广大用户对我们的支持和厚爱,我们将秉持初心,勇攀高峰,我们的“蜗壳365”将持续提供极致的免费服务。
